% Chapter 4

\newglossaryentry{pip}{name=pip, description={Pip Installs Packages is a package management system used to install and manage software packages written in Python}}

\chapter{Installation guide} % Write in your own chapter title
\label{chap:installation}
\lhead{Chapter 4. \emph{Installation}} % Write in your own chapter title to set the page header
\begin{flushright}
\textit{``If Microsoft ever does applications for Linux it means I've won.''} \\ Linus Torvalds
\end{flushright}


In this chapter, the complete installation process of the developed script will be presented.

\section {Setting up the environment}

In order to use the developed tool, it is highly recommended to install it on a system providing a GNU/Linux distribution. The tool might work under Windows or MacOS as the used libraries should all be cross-platform, but the software has never been tested under these platforms.

As the main used language is Python you should install it with the following command :


\section{Use the packaged version}
In order to simplify the installation process, a packaged version has been built and is ready to download on the project's \href{https://github.com/dchenaux/Yoda}{GitHub page}.

The installation process is really straightforward and since it's a \gls{pip} package.

\section{From source code}

\section{Deploy on Apache}

\subsection{Accessing the user interface}
Assuming the installation process has already been followed as explained in the chapter 4, the user interface should be accessible in any browser. When using the flask integrated web server, the application is available locally under the adress \url{http://127.0.0.1:5000/}. If installed on an external webserver it depends on the configuration of this last one. When configured as in the installation guide it should be accessible locally through \url{http://127.0.0.1:80}

