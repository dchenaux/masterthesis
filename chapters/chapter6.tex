\chapter{Conclusion} % Write in your own chapter title
\label{chap:conclusion}
\lhead{Chapter 6. \emph{Conclusion}} % Write in your own chapter title to set the page header

In this final chapter, we summarize the work done in this master thesis and give complementary remarks along with potential improvements that came into mind during the development and then the writing of the report.

\section{Final remarks}
The main goal of the thesis was to develop a system which could pursue effective code maintenance with continuous data collection. This report is the summary of six months of development and in here we additionally presented related works in order to build a theoretical background. Our proposed solution is explained in a dedicated chapter along with a handy installation guide. The explanations were focused on the core features of our system because we did not want to overwhelm the reader with too much technical aspects. Indeed, even if the coding of the user interface was around half of the coding time, we only presented the final features without spending time about its programming implementation.  Additionally, we conducted some experiments on the developed applications to find out if it engendered some drawbacks. During the development, we thought also about some future work which could not be included in the scope of a master thesis. These points are exposed in the next section.

\section{Personal remarks}

On a personal point of view, this master thesis was the second experience in the development of a complete high technical subject. But even though, I had already the chance to develop a bin packing system for my Bachelor Thesis, this was a completely new challenge in a absolutely unfamiliar field. As the development was cut in two half, because I was in Austria for an internship during the summer, I also lost some time to step back into the code after this long pause. On the other hand, this brought a fresh view on the code and some parts that seemed really good before were rewritten in a more clever way.

Also, the use of Python was a new challenge for me as I was more used to develop in PHP for this kind of application and therefore I had to learn to know and use all the different libraries. The choice of MongoDB was also a discovery as I was more used to work with relational databases.

Finally, to conclude this work, I learned the hard way that programmers may plan as well as they want the available time, but there will always be unexpected events. Particularly on this kind of research field, it was quite difficult for me to estimate how much time could be needed to develop an idea. I can clearly remember that after two months of development I already had the felling that I was close to the end. This is something I want to take with me for my future projects.

\section{Future work}
As the developed application is only a proof-to-concept system, there is plenty of room for future improvements. In this final section, we want to propose the reader some topics and features that came into mind during the development and the final experiments. These improvements could lead to a software which could be used by final users. Following, in order of importance, some of the further work we could think about:
\begin{itemize}
  \item At the time, the data capture model has some issues with complex data retrieval. Indeed, it is more than likely that the system will throw an error when the analyzed script opens large text files in order to work with them. Our insight is that there is a conflict between the JSON way of the MongoDB database and the pythonic way of storing dictionaries;
  \item Currently, the system was focused on numeric variables, even if it is capable of retrieving strings. We would suggest some enhancement in that way by first removing the string variables from the plot charts and then perhaps find an another way to represent them. One idea could be to create a kind of tree mapping of the different values;
  \item The plot graphs definitely need a zoom functionality. In deed, we observed that with scripts which can generate a big amount of values it is increasingly hard to use the graphs; 
  \item The following point is something we already discussed at the beginning of the coding phase and it is still an open point. In the actual proposed solution, all the values are stored in the database. In the example of a looping variable which could range from 1 to 1 million, is it really useful to store every value in between? We think that there could be a simpler way to store such data;
  \item A last point concerning the user interface itself, thanks to some new framework like \textit{Electron}, it would be pretty easy to adapt it to a standalone application.
\end{itemize}
  
