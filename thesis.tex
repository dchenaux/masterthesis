%% ----------------------------------------------------------------
%% Thesis.tex -- MAIN FILE (the one that you compile with LaTeX)
%% ---------------------------------------------------------------- 

%% Manual stuff todo at the end:
% - put the URL addresses in the extra field of the bibtex file in \url{}
% - delete the howpublished items from bibtex file
% - identicon upper/lower case

% Set up the document
\documentclass[a4paper, 11pt, oneside]{thesis}  % Use the "Thesis" style, based on the ECS Thesis style by Steve Gunn
\graphicspath{{figures/}}  % Location of the graphics files (set up for graphics to be in PDF format)

% Microtype
\usepackage[activate={true,nocompatibility},final,kerning=true,spacing=true,factor=1100,stretch=10,shrink=10]{microtype}
% activate={true,nocompatibility} - activate protrusion and expansion
% final - enable microtype; use "draft" to disable
% tracking=true, kerning=true, spacing=true - activate these techniques
% factor=1100 - add 10% to the protrusion amount (default is 1000)
% stretch=10, shrink=10 - reduce stretchability/shrinkability (default is 20/20)

% Include any extra LaTeX packages required
\usepackage{makeidx}
\usepackage[utf8]{inputenc}
\usepackage[english]{babel}
\usepackage{natbib}  % , sort&compress, sort&compress,authoryear, square, comma] Use the "Natbib" style for the references in the Bibliography
\usepackage{verbatim}  % Needed for the "comment" environment to make LaTeX comments
\usepackage{vector}  % Allows "\bvec{}" and "\buvec{}" for "blackboard" style bold vectors in maths
\usepackage[usenames,dvipsnames]{color}
\usepackage{setspace}
\definecolor{DarkGray}{RGB}{90,90,90}
\definecolor{DarkDarkBlue}{RGB}{0,40,120}
\definecolor{DarkBlue}{RGB}{40,80,150}
\definecolor{DarkGreen}{RGB}{80,150,40}
\definecolor{DarkRed}{RGB}{140,50,80}


\usepackage[]{listings}
\usepackage{pdfpages}
\usepackage{varwidth}
\usepackage[nopostdot,nonumberlist,numberedsection]{glossaries}
\usepackage{tabulary}
\usepackage{tabularx}
\usepackage{multirow}
\usepackage{pdfpages}

%--- Python code highlighting ---
%http://tex.stackexchange.com/questions/83882/how-to-highlight-python-syntax-in-latex-listings-lstinputlistings-command#83883

  \usepackage[utf8]{inputenc}

  % Default fixed font does not support bold face
  \DeclareFixedFont{\ttb}{T1}{txtt}{bx}{n}{10} % for bold
  \DeclareFixedFont{\ttm}{T1}{txtt}{m}{n}{10}  % for normal

  % Custom colors
  \usepackage{color}
  \definecolor{deepblue}{rgb}{0,0,0.5}
  \definecolor{deepred}{rgb}{0.6,0,0}
  \definecolor{deepgreen}{rgb}{0,0.5,0}
  \definecolor{mygray}{rgb}{0.5,0.5,0.5}

  \usepackage{listings}

  % Python style for highlighting
  \newcommand\pythonstyle{\lstset{
  language=Python,
  basicstyle=\ttm,
  otherkeywords={self},             % Add keywords here
  keywordstyle=\ttb\color{deepblue},
  emph={MyClass,__init__},          % Custom highlighting
  emphstyle=\ttb\color{deepred},    % Custom highlighting style
  stringstyle=\color{deepgreen},
  frame=L,                         % Any extra options here
  showstringspaces=false,
  breaklines=true,
  numbers=left,                     
  numberstyle=\ttb\color{mygray},
  numbersep=20pt,
  xleftmargin=25pt,
  framesep=10pt,          
  }}


  % Python environment
  \lstnewenvironment{python}[1][]
  {
  \pythonstyle
  \lstset{#1}
  }
  {}

  % Python for external files
  \newcommand\pythonexternal[2][]{{
  \pythonstyle
  \lstinputlisting[#1]{#2}}}

  % Python for inline
  \newcommand\pythoninline[1]{{\pythonstyle\lstinline!#1!}}
  
%--- End of Python Highlighting --



\hypersetup{urlcolor=DarkDarkBlue, colorlinks=true}  % Colours hyperlinks in blue, but this can be distracting if there are many links.
\setcounter{tocdepth}{2}
\makeindex
\makeglossaries

%% ----------------------------------------------------------------
\begin{document}

\frontmatter	  % Begin Roman style (i, ii, iii, iv...) page numbering
\pagenumbering{arabic}
% Set up the Title Page
\title{Effective code maintenance with continuous data collection}
\authors  {\texorpdfstring
            {\href{mailto:david.chenaux@unifr.ch}{David Chenaux}}
            {David Chenaux}
            }
\addresses  {\groupname\\\deptname\\\univname}  % Do not change this here, instead these must be set in the "Thesis.cls" file, please look through it instead
\date       {\today}
\subject    {}
\keywords   {}

\def\chapterautorefname{Chapter}%
\def\sectionautorefname{Section}%
\def\subsectionautorefname{Subsection}%


%% define special words
\newcommand{\idname}{Moji-Identicon} 
\newcommand{\idnames}{Moji-Identicons} 


\maketitle
%% ----------------------------------------------------------------

\setstretch{1.3}  % It is better to have smaller font and larger line spacing than the other way round

% Define the page headers using the FancyHdr package and set up for one-sided printing
\fancyhead{}  % Clears all page headers and footers
\rhead{\thepage}  % Sets the right side header to show the page number
\lhead{}  % Clears the left side page header

\pagestyle{fancy}  % Finally, use the "fancy" page style to implement the FancyHdr headers

%% ----------------------------------------------------------------

% The Abstract Page
\addtotoc{Abstract}  % Add the "Abstract" page entry to the Contents
\abstract{
\addtocontents{toc}{\vspace{1em}}  % Add a gap in the Contents, for aesthetics
\setcounter{page}{2}
Code analysis has been around for a few decades already, yet it is mostly limited to static analysis without considering the actual data stream flowing through the code. The intention of this thesis is to implement a proof-of-concept system for effective code maintenance with continuous data collection.

The results of the development are detailed in the present report. In order to understand the field of Dynamic Program Analysis, this work first dedicate an entire chapter about related work which includes a definition of the program analysis, several approaches and a brief overview of available tools. Then, a presentation of the coded system is made and finally we propose at the end of the thesis some future work which could be implemented in the developed system.


}

\clearpage  % Abstract ended, start a new page
%% ----------------------------------------------------------------
% The "Funny Quote Page"
\pagestyle{empty}  % No headers or footers for the following pages

\null\vfill
% Now comes the "Funny Quote", written in italics
%\textit{``Another nice quote will be found to be here!!!''}

\begin{flushright}
%Lorem van Ipsum, The book\citep{wilkinson2005}
\end{flushright}
%\clearpage  % Funny Quote page ended, start a new page
%% ----------------------------------------------------------------

\setstretch{1.3}  % Reset the line-spacing to 1.3 for body text (if it has changed)

% The Acknowledgements page, for thanking everyone
\acknowledgements{
\addtocontents{toc}{\vspace{1em}}  % Add a gap in the Contents, for aesthetics

First, I would like to thank Prof. Philippe Cudré-Mauroux for the creation and the lead of the eXascale Infolab. This research group is a great asset for the  University of Fribourg and gives amazing opportunities to explore technical and experimental fields in many informatic sectors.

Then, I would particulary like to thank Roman Prokofyev who supervised my work and who has been really supportive and comprehensive with my time schedule. I thank as well Michael Luggen who came in backup and gave a fresh insight on the developed proof-of-concept system.

Finally, I send all my gratitude to Prof. Marino Widmer for helping me with my administrative struggles and Gaëtan Vieux for his final corrections and suggestions.
}
\vfill\vfill\null
\clearpage  % End of the Acknowledgements
%% ----------------------------------------------------------------


\pagestyle{fancy}  %The page style headers have been "empty" all this time, now use the "fancy" headers as defined before to bring them back

%% ----------------------------------------------------------------
\lhead{\emph{Contents}}  % Set the left side page header to "Contents"
\tableofcontents  % Write out the Table of Contents

%% ----------------------------------------------------------------
\lhead{\emph{List of Figures}}  % Set the left side page header to "List if Figures"
\listoffigures  % Write out the List of Figures
\clearpage  % End of the Acknowledgements

%% ----------------------------------------------------------------
\lhead{\emph{List of Tables}}  % Set the left side page header to "List of Tables"
\listoftables  % Write out the List of Tables

%% ----------------------------------------------------------------
% End of the pre-able, contents and lists of things
% Begin the Dedication page

\setstretch{1.3}  % Return the line spacing back to 1.3

%\pagestyle{empty}  % Page style needs to be empty for this page
%\dedicatory{For/Dedicated to/To my\ldots}

\addtocontents{toc}{\vspace{2em}}  % Add a gap in the Contents, for aesthetics


%% ----------------------------------------------------------------
\mainmatter	  % Begin normal, numeric (1,2,3...) page numbering
\pagestyle{fancy}  % Return the page headers back to the "fancy" style
% Include the chapters of the thesis, as separate files
% Just uncomment the lines as you write the chapters
\setcounter{page}{8}

% Chapter 1
\newglossaryentry{dpa}{name=DPA, description={Dynamic Program Analysis}}
\newglossaryentry{spa}{name=SPA, description={Static Program Analysis}}
\newglossaryentry{ide}{name=IDE, description={Integrated development environments}}

\chapter{Introduction} % Write in your own chapter title
\label{chap:introduction}
\lhead{Chapter 1. \emph{Introduction}} % Write in your own chapter title to set the page header
Code analysis has been around for a few decades already, yet it is mostly limited to static analysis without considering the actual data stream flowing through the code. Ability to display the actual data running through the system promises many potential benefits, including easier debugging and code recall, which results in significantly lower code maintenance costs. 

\section{Problem definition}
Every developer is more or less feared about the debugging and code reviewing phase of their software. Obviously, this process can sometimes take several painfully hours and each programmer knows how frustrating it can be to search for a hidden bug in thousands lines of codes. In order to support the programmers in this hated task, debuggers are the most useful existing tools which are part of the so called \textit{static program analysis} (\gls{spa}). 

With the apparition of object-oriented programming language, searching for syntactic errors in the code is not anymore sufficient. Therefore, a new research field was pushed forward which is called the \textit{dynamic program analysis} and consists in analyzing the software during its execution. This procedure allows to take in account some possible inputs which were not probed with the SPA. Yet none of the modern code analysis tools was able to successfully integrate their source code editors with the actual data stream flowing through the code.

\section{Objectives}
The goal of this project is to design a proof-of-concept system in one programming language that allows full code instrumentation. This system should be able to :
\begin{itemize}
   \item seamlessly capture some or all values for all variables in source code and store them somewhere ;
   \item easily retrieve the saved values by providing an API to the storage in order to make the data accessible for navigation in third-party applications;
   \item provide a basic visualizing interfacein order to allow an easy analysis of the results.  
 \end{itemize}  
 Finally, an evaluation of system's performances will be established through different experiments.

\section{Organization}
The thesis is organized into four sections :
\begin{enumerate}
  \item \textbf{Related work}: In this first chapter of the thesis, we present an insight of the existing work on the field \textit{program analysis} and in particular the \gls{dpa}. This is including a definition of the field and its particularities, an overview of some available solutions side by side with the current restrictions.
  \item \textbf{Development}: This part describes on the architecture of the proof-to-concept system with a presentation of the proposed solution and detailed information about its structure.
  \item \textbf{Installation guide}: An installation guide of the software which describes the required environment, the package installation and the compilation of the system.
  \item \textbf{Experiments}: In this section, we conducteur a number of experiments in order to test and check the performance and results of the software.
  \item \textbf{Conclusions}.
\end{enumerate}
The thesis concludes with some outputs and is proposing some future improvements which could be relevant.


% Chapter 2
\newglossaryentry{vm}{name=VM, description={Virtual Machine}}
\newglossaryentry{jpda}{name=JPDA, description={Java Platform Debugger Architecture}}
\newglossaryentry{sdk}{name=SDK, description={Software Development Kit}}
\newglossaryentry{pdb}{name=PDB, description={The Python Debugger}}
\newglossaryentry{aop}{name=AOP, description={Aspect Oriented Programming}}
\newglossaryentry{libre}{name=Libre, description={or Free software, is distributed under terms that allow users to run the software for any purpose as well as to study, change, and distribute the software and any adapted versions.}}
\newglossaryentry{jvmpi}{name=JVMPI, description={Java Virtual Machine Profiling Interface}}
\newglossaryentry{jvmti}{name=JVMTI, description={Java Virtual Machine Tools Interface}}
\newglossaryentry{smt}{name=SMT, description={Satisfiability Modulo Theories}}
\newglossaryentry{ast}{name=AST, description={Abstract Syntax Tree}}

\chapter{Related work}
\label{chap:relatedwork}
\lhead{Chapter 2. \emph{Related work}} % Write in your own chapter title to set the page header
\begin{flushright}
\textit{``Sharing is good, and with digital technology, sharing is easy.''} \\ Richard Stallman
\end{flushright}

The intention of this thesis, as brought up in the introduction, would be to implement a dynamical program analysis system. In order to meet this goal, it is necessary to build a theoretical understanding of "Program Analysis" and therefore the present chapter will endeavor to do a presentation of the subject. The first part propose a definition of the field, then the second suggest several technical approaches. Following, the third section introduces some popular analysis tools, to finally discuss the actual limitations of dynamic analysis in the fourth part.

\section{What is Program Analysis ?} 
Programming environments are an essential key for the acceptance and success of a programming language. After \cite{Ducasse1994}, without the appropriate developments and maintenance tools, programmers are likely to have a bad software understanding and therefore produce low-quality code. They will be therefore reluctant to use a language without appropriate programming environments, however powerful the programming language is.

As already introduced in the previous chapter, program analysis is an automated process which aims to analyze the behavior of a software regarding a property such as correctness, robustness, safety and liveness. Program analysis can be separated in two methods : the \gls{spa} which is performed without actually executing the software and the \gls{dpa} which is performed during the runtime. \citep{Wikipedi2016}

The \gls{spa} is a straightforward solution because it does not require running the program for analyzing its dynamic behavior. The analysis consists in going through the source code and highlights coding errors or ensure conformance to coding guidelines. A classic example of static analysis would be a compiler which is capable of finding lexical, syntactic and even semantic mistakes. The main advantage of this method is that it allows to reason about all possible executions of a program and gives assurance about any execution, prior to deployment. 

Nevertheless, according to \cite{Gosain2015}, since the widespread use of object oriented languages, \gls{spa} is found to be ineffective. This can be explained due to the usage of runtime features such as dynamic binding, polymorphism, threads etc. To remedy this situation, developers call on \gls{dpa} which can, after \cite{Marek2015100}, allows to gain insight into the dynamics and runtime behavior of those systems during execution. Moreover, because the runtime behavior depends now on many other factors, such as program inputs, concurrency, scheduling decisions, or availability of resources, static analysis does not allow full understanding of the code. The following table, proposed by \cite{Gosain2015}, is resuming the main differences between static and dynamic analysis.

\bigskip

\begin{table}[htb]
\begin{center}
\begin{tabulary}{\textwidth}{CC}
  \hline
  Dynamic Analysis 	& Static Analysis \\\hline
  Requires program to be executed	& Does not require program to be executed \\
  More precise & Less precise \\
  Holds for a particular execution & Holds for all the executions \\
  Best suited to handle runtime programming lan- & Lacks in handling runtime programming lan-\\
guage features like polymorphism, dynamic bind & guage features.\\
ing, threads etc. &  \\
  Incurs large runtime overheads & Incurs less overheads \\\hline
\end{tabulary}
\end{center}
\caption{Comparison of Dynamic analysis with Static Analysis}
\label{list:comparaison}
\end{table}

\bigskip

In the light of this comparison, it is well worth noting that Dynamic Program Analysis does not substitute the Static Analysis. Quite the reverse, both are interdependent tools and even if Static Program Analysis is not sufficient anymore, it still gives relevant information about the code to the programmer. The DPA should come in a second phase when the source code has been validated through SPA. As it can be surmised, the ability to examine the actual and exact runtime behavior of the program might be the DPA main advantage, whereas SPA prime edge could be the independence of input stimuli and the generalization for all executions. To illustrate these characteristics, some program analysis solutions are presented further in this chapter.

\pagebreak

\section{Program Analysis approaches}
Now that a definition of Program Analysis has been established, we are going to explore some different approaches for going into the subject in depth. Yet, since the field is expansive, the purpose of this section is not to cover the entire subject. The reading of this section should, notwithstanding, give a good overview to the reader. Here we describe, first, the essential static analysis methods followed in a second time with the dynamic analysis techniques.

\subsection{Static methods}

The static methods are regrouped in four different categories proposed by \cite{Nielson2004} and briefly presented here, some information was also gathered from the \cite{Wikipedi2016} page which is proposing a grouping based on the same criteria.

\begin{description}
  \item[Data Flow Analysis] is a technique which consist in gathering information about the values and their evolution at each point of the program. In the Data Flow Analysis the program is considered as a graph in which the nodes are the elementary blocks and the edges describe how control might pass from one elementary block to another.
  
  \item[Constrained Based Analysis] or Control Flow Analysis, intent to know which functions can be called at various points during the execution ; what "elementary blocks" may lead to what other "elementary blocks".
  
  \item[Abstract Interpretation] resides in proving that the program semantics satisfies its specification according to \cite{Cousot2008}. What the program executions actually do should satisfy what the program executions are supposed to do. It can be sumarized as a partial execution of a program which gather information about its semantics without performing all the calculations.
  
  \item[Type and Effect Systems] are two similar techniques where the second one can be seen as an extension of the first. Type systems are using types, which are a concise, formal description of the behavior of a program fragment. \cite{Remy2017} explains that programs must behave as prescribed by their types. Hence, types must be checked and ill-typed programs must be rejected. Effect systems are, after \cite{Nielson1999}, an extension of annotated type system where the typing judgments take the form of a combination of a type and an effect. This combination is associated with a program relative to a type environment.
\end{description}


\subsection{Dynamic methods}

In the past section, some static analysis methods have been defined and therefore, the dynamic methods are depicted here. As it was heretofore specified, dynamic analysis is a quite recent research field which status could be still defined as academical. Naturally, the different techniques are not as well established as for the static analysis and can vary a lot in accordance with the author of the different papers. For this work, the following particular methods were privileged and were already proposed by \cite{Gosain2015} in their survey of Dynamic Program Analysis Techniques and Tools. 

\begin{description}
  \item[Instrumentation based approach] needs a code instrumenter used as a pre-processor in order to inject instrumentation code into the target program. This can be done at three different stages : source code, binary code and bytecode. The first stage adds instrumentation code before the program is compiled, the second one adds it by modifying or re-writing compiled code and the last one performs tracing within the compiled code.
  
  \item[\gls{vm} Profiling based technique] uses the profiling and debugging mechanism provided by the particular virtual machine, for example the \gls{jpda} for Java \gls{sdk} or the \gls{pdb} for Python. These profilers give an insight into the inner operations of a program, especially the memory and heap usage. To capture this profiling information plug-ins are available and can access the profiling services of the VM. Benchmarks are then used for actual runtime analysis which acts like a black-box test for a program. This process involves executing or simulating the behavior of the program while collecting data which is reflecting the performance. Unfortunately this technique has the drawback of generating high runtime overheads. 
  
  \item[Aspect Oriented Programming] aims to increase modularity by allowing the separation of cross-cutting concerns. Because there is no need to add instrumentation code as the instrumentation facility is integrated within the programming language, the additional behavior is added to existing code without modifying the code itself. \gls{aop} adds the following constructs to a program : aspects, join-point, point-cuts and advices. These constructs can be considered like classes. Most popular languages have their aspect oriented extensions like AspectC++ and AspectJ. In Python, there are some libraries which aim to reproduce AOP behavior but there isn't any canonical one. Actually there is a debate to what extent aspect oriented practices are useful or applicable to Python's dynamic nature. %
  
\end{description}


\section{Program Analysis tools}
As the theoretical background is now settled, we want to propose in this section some static and dynamic analysis tools. The reader will discover in the next chapter that the proof-to-concept system is coded in Python and therefore additional information is given here for solutions available in that language.

\subsection{Static Analysis tools}
Following, some of the most popular tools (commercial or free) for SPA are described, picked in widespread languages : Java, C/C++ and Python. The diverse description are summarized versions of the \cite{Gomes2009} paper along with some official information gathered on the tools websites and their respective Wikipedia pages.

Starting with C/C++, \textbf{Splint} is a very well known tool, allowing to check for security vulnerabilities and coding mistakes. Splint is based on Lint and tries to minimize the efforts needed for its deployment. Additionally, with some annotation, Splint can extend its performances over Lint. Splint can among others detect : dereferencing a possibly null pointer, memory management errors including uses of dangling references and memory leaks, problematic control flow such as likely infinite loops. \textbf{Astrée} is based on abstract interpretation and can analyze safety-critical applications written or generated in C. It proves the absence of run­time errors and invalid concurrent behavior for embedded applications as found in aeronautics, earth transportation, medical instrumentation, nuclear energy, and space flight. Another worth mentioning tool is the \textbf{PolySpace Verifer} tool developed by MathWorks who also created the famous Matlab software. 

Concerning Java, one recognized tool is Findbugs. With the advantage of being a \gls{libre} software, the application uses a series of ad-hoc techniques designed to balance precision, efficiency and usability. FindBugs operates on Java bytecode, rather than source code. Another Libre software is \textbf{Checkstyle} which, as his name gives a hint, allows to report any breach of standards in the source code. Finally a commercial tool, \textbf{Jtest} which is an integrated Development Testing solution, can perform Data-flow analysis Unit test-case generation and execution, static analysis, regression testing, runtime error detection, code review, and design by contract.

In the Python world, \textbf{Pylint} is a coding standard checker which follows the style recommended by the PEP 8 specification. It is also capable of detecting coding errors and is integrable in IDEs. Speaking of IDEs, \textbf{PyCharm} includes also static analysis functions like PEP8 checks, testing assistance, smart refactorings, and a host of inspections.

\subsection{Dynamic Analysis tools}

Just as for the static tools, the most popular DPA software are presented here. Following, a table proposed by \cite{Gosain2015} with a summary of some available DPA tools regrouped by technique. The table indicates the concerned language and also which type of dynamic Analysis is performed by the application.
\begin{table}[htb]
\begin{center}
\begin{tabulary}{\textwidth}{L|L|L|C|C|C|C|C|C|C|C|C}
\hline
Technique             & Tool                      & Language             & \multicolumn{9}{c}{Type of Dynamic Analysis done}\\   
\hline
  & & & \rotatebox{90}{Cache Modelling} & \rotatebox{90}{Heap Allocation} & \rotatebox{90}{Buffer Overflow} & \rotatebox{90}{Memory Leak} & \rotatebox{90}{Deadlock Detection} & \rotatebox{90}{Race Detection} & \rotatebox{90}{Object LifeTime} & \rotatebox{90}{Metric Computation} & \rotatebox{90}{Invariant Detection} \\ 
\hline
                      & Daikon                    & C,C++                & & & & & & & & &\checkmark \\
                      & Valgrind                  & C,C++                & & & &\checkmark& &\checkmark& & &\\
Instr.Based           & Rational Purify           & {\tiny C, C++, Java} & & & &\checkmark& & & & & \\
                      & {\tiny Parasoft Insure++} & C,C++                & &\checkmark& &\checkmark & & & & & \\
                      & Pin                       & C                    &\checkmark & & & & & & & \\
                      & Javana                    & Java                 &\checkmark& & & & & &\checkmark & &  \\
                      & DIDUCE                    & Java                 & & & & & & & & &\checkmark  \\
\hline
AOP Based             & DJProf                    & Java                 & &\checkmark& & & & &\checkmark& & \\
                      & Racer                     & Java                 & & & & & &\checkmark& & & \\
\hline
                      & Caffeine                  & Java                 & & & & & & &\checkmark&& \\
VM Profiling          & DynaMetrics               & Java                 & & & & & & & &\checkmark& \\
Based                 & *J                        & Java                 & & & & & & & &\checkmark& \\
                      & JInsight                  & Java                 & & & &\checkmark&\checkmark& &\checkmark& & \\
\hline
\end{tabulary}
\end{center}
\caption{Dynamic Analysis Tools}
\label{list:dynamictools}
\end{table}

\textbf{Valgrind, Purify and Insure++} are instrumentation based, and can automatically detect memory management and threading bugs among with profiling a program in details. While Valgrind is a instrumentation framework for building dynamic analysis tools, the two others are fully-fledged analysis software. \textbf{Javana} comes with an easy-to-use instrumentation framework so that only a few lines of instrumentation code have to be programmed for building powerful profiling tools. \textbf{Daikon} and \textbf{Diduce} are trendy tools for invariant detection and are respectively an offline and online tool. Last but not least, \textbf{Pin} is a dynamic binary instrumentation framework developed by Intel. It enables the creation of dynamic program analysis tools and can be used to observe low level events like memory references, instruction execution, and control flow as well as higher level abstractions such as procedure invocations, shared library loading, thread creation and system call execution.

For AOP based applications, the two selected programs are \textbf{DjProf} and \textbf{Racer}. The first one is a profiler used for the analysis of heap usage and object life-time analysis and the second one is a data race detector tool for concurrent programs.  

\textbf{*J} and \textbf{DynaMetrics} are two academical research projects about Virtual Machine profiling and are proposing a solution for computing dynamic metrics for Java. The first one, proposed by \cite{Dufour2003}, relies on \gls{jvmpi}, while the second solution, from \cite{Singh2013}, relies on the new \gls{jvmti}. \textbf{JInsight} is for exploring visually runtime behaviour of Java programs and \textbf{Caffeine} helps to check conjectures about Java programs.

In addition to this table, some Python tools are also available even if the field seems not to be really well developed for this programming language. This could be explainable because of the dynamic nature of the language and might be why the following tools are developed \textit{in} Python but not \textit{for} it. The first tool is \textbf{Angr} which is a Python framework for analyzing binaries. It focuses on both static and dynamic instrumentation analysis, making it applicable to a variety of tasks. \textbf{Triton} is another binaries analyzer framework and proposes python bindings. Its main components are Dynamic Symbolic Execution engine, a Taint Engine, \gls{ast} representations of the x86 and the x86-64 instructions set semantics, \gls{smt} simplification passes, an SMT Solver Interface 


\section{Dynamic Analysis limitations}

DPA is a quite new research field and as a consequence induces ineluctably some drawbacks and limitations. The following table created by \cite{Gosain2015} gives a good overview of the different techniques and some of their drawbacks.

\begin{table}[htb]
\begin{center}
\begin{tabulary}{\textwidth}{L|LLLL}
\hline
  & \multicolumn{2}{c}{Instrumentation} & VM Profiling & AOP\\
  & Static & Dynamic\\
\hline
Level of Abstraction      & Instruction/Bytecode  & Instruction/Bytecode  & Bytecode      & Programming Language\\
\hline
Overhead                  & Runtime               & Runtime               & Runtime       & Design and deployment\\
\hline
Implementation Complexity & Comparatively low     & High                  & High          & Low\\
\hline
User Expertise            & Low                   & High                  & Low           & High\\
\hline
Re-compilation            & Required              & Not Required          & Not Required  & Required\\  
\hline
\end{tabulary}
\end{center}
\caption{Dynamic Analysis Techniques comparison}
\label{list:limitations}
\end{table}

The \autoref{list:limitations} shows straightforwardly some limitations of the different Dynamic Analysis techniques. Instrumentation and VM Profiling based techniques engender high runtime overheads whereas AOP rises heavy design and deployment efforts. While the implementation complexity is rather high for Dynamic Instrumentation and VM Profiling, a strong user expertise is also needed for the first one. Finally recompilation is required for two on four techniques.

Additionally, the programmer must be aware that the automated tools cannot guarantee the full test coverage of the source code. Moreover, however how powerful the tools can be, they might yet produce false positives and false negatives. This is why a human code understanding and reviewing is still an absolute necessity.

\section{Concluding remarks}

In this chapter, we summarized related work about program analysis. After defining what program analysis is, we briefly presented some of the static and dynamic approaches with their respective techniques. However, this is by no means an exhaustive presentation of all the approaches and the reader must be aware that the field is far more complex than that. 

To complete these theoretical explanations, we presented some popular tools for both approaches and discussed some general dynamic analysis limitations. During the redaction of the chapter, it appeared clearly that the DPA field is quite recent and is at the moment being actively researched. 

In the next chapter, we will introduce our own contribution with  development of the proof-to-concept system.


% Chapter 3

\newglossaryentry{json}{name=JSON, description={JavaScript Object Notation}}

\chapter{Development} % Write in your own chapter title
\label{chap:development}
\lhead{Chapter 3. \emph{Development}} % Write in your own chapter title to set the page header
\begin{flushright}
\textit{``For me, open source is a moral thing.''} \\ Matt Mullenweg
\end{flushright}

In this chapter, we introduce our contribution to the dynamic program analysis. As explained in the introduction the aim is to develop a proof-to-concept system and all the steps to achieve it will be presented in details including the Setup, Data capture model, Data model and its user interface.


\section{Proposed solution}
While working on a growing project, there is always a point where it becomes difficult to keep an eye on all the variables. In order to give the programmer an overview of the variables evolution, this work is aiming to propose a proof-to-concept system which will not only monitor the data evolution, but also give the possibility to compare the gathered data between different runs.

To achieve such a system, the project is going to be separated in three different parts which will constitute the system. First, a data capture model will monitor all the needed variables and their evolution during the execution of the reviewed program. Then a data model will be created and backup procedure will be implemented to store the data in this model. Finally, a web-application will process the extracted data and show them for reviewing the results. Each mentioned part is exposed in the development section.

\section{Environment}
For the project 4 main technologies were chosen in order to develop the required features. 

First the data is captured in \textit{Python} with the help of the integrated Debugger Framework. Python is a widely used high level programming language which has seen these last year an increasing enthusiasm around it, especially for web based applications. Thanks to the dynamic nature of Python, which includes a dynamic type system, the real-time collection of object is a pretty straightforward process and therefore it made plenty sense to use it in our project. More over Python offers good compatibility with other programming language since there are a lot of bindings available. For this project, the newest version 3 of the programming language was chosen because of the better handling of encoding.

Secondly, the extracted data is stored in a \textit{MongoDB} Database. MongoDB is a document-oriented database which enters in the new categorie of No-SQL database systems. MongoDB has the advantage to use \gls{json} like documents with schemas which was a clever choice to store the heterogeneous extracted data.

Finally, the used interface was built with the help of \textit{Python}, \textit{Html/CSS} and \textit{Javascript}. As already said, Python is now an interesting language to develop web applications and was used here, with the help of the Flask framework, for the server side process. HTML/CSS and Javascript were used for the presentation of the results.

Additionally, the chosen IDE was PyCharm academic edition version 2015 and then 2016. PyCharm is a very complete IDE which supports among others Python web frameworks, database support, code inspection. In order to optimize the development management, the GitHub online tool was chosen as version control repository. The deployment during the development of the solution was tested on virtual machine server under Ubuntu Server 14.04. The server is provided by the Department of Informatics at the University of Fribourg and is accessible internally at \url{http://diufpc115.unifr.ch/}. This server will also be used for the experiments in the chapter 5. 

In order to deploy regularly the newest version of the ongoing work an automation server named Jenkins was configured. Jenkins was charged to fetch every day the latest prototype on the GitHub repository, create a package of it and install it on the server. If during this process a bug occurred, an e-mail to the interested persons was sent.

In the next section, each module of the proposed system will be exposed in details regarding their functionality and their implementations.

\section{Data Capture Model}
This section is presenting the development phase of the data capture model. The data capture model, or \textit{analyser} as it was called during the development, is the core of the system and is based on the Python Debugger Framework (BDB). BDB handles basic debugger functions, like setting breakpoints or managing execution. Thanks to the object-oriented programming, the classes and the function inheritance, it is a straightforward job to rewrite the different functionality as needed for this project. 

The development began with study of a script provided by Roman Prokofyev which is implementing some basic data capture functionality derived from the Python debugger framework. The understanding of the developed concepts was the first step to the creation of the data capture model. The analyser consists in 240 lines of code and some of the most important functions are explained here.

\subsection{Setting up the trace}
In order to use the analyzer, some code has to be added at the beginning of the aimed file. The code is necessary to import the module and to set the start of the tracing phase. 
\begin{python}
import yoda.analyser
yoda.analyser.db.set_trace()
\end{python}

The \pythoninline{set_trace()} function is inherited by the BDB and is needed to start debugging with a Bdb instance from caller’s frame.
It is also absolutely necessary to stop the trace at the end of the aimed code with the \pythoninline{yoda.analyser.db.set_quit()
} function which set the quitting attribute to \texttt{True}. This raises BdbQuit in the next call to one of the \pythoninline{dispatch_*()} methods. 

For further information about the operating of the Python Debugger Framework, we advise the reader to refer to the official documentation \citep{Foundation2017}.

\subsection{Initialization}

Once the analyser module called, the first step is to setup the \pythoninline{Yoda} class along with some global variables needed during the tracing process. The first variable \pythoninline{json\_results} (line 2) will be explained further but is basically where the extracted data will be stored. Then, the \pythoninline{instrumented\_types} list (line 3) limits the instrumented objects to this list, it is possible to add further objects if needed. The next 6 variables (line 4-9) are needed for gathering and computing line numbers, frames and files name. Finally, the \pythoninline{next_backup} variable (line 10) defines a limit of how many lines can be analyzed before flushing the information in the database.

\begin{python}
class Yoda(bdb.Bdb):
    json_results = None
    instrumented_types = (int, float, str, list, dict)
    prev_lineno = defaultdict(int)
    prev_lineno['<module>'] = 0 
    cur_framename = '<module>'
    file_name = None
    file_id = None 
    total_linenb = 0
    next_backup = 1000
\end{python}

As the needed variables are now set up, the script continues with the initialization of the \pythoninline{Yoda} class. Within the class, the connection of the database is also created if case of production mode (line 4). 

\begin{python}
def __init__(self):
    bdb.Bdb.__init__(self)
    if settings.DEBUG is False:
        mongoengine.connect(settings.MONGODB)
\end{python}

\subsection{Event Catching}

BDB can react to various events during the code execution which are handled by 4 functions : \texttt{user\_call, user\_line, user\_return, user\_exception}. Each function has been rewritten in order to redirect the event to a self-written handling function called \texttt{interaction}. 
\smallskip
\begin{python}
def user_call(self, frame, args):
    self.interaction(frame, 'call', None)
def user_line(self, frame):
    self.interaction(frame, 'line', None)
def user_return(self, frame, value):
    self.interaction(frame, 'return', None)
def user_exception(self, frame, exception):
    self.interaction(frame, 'exception', exception)
\end{python}

Once the \texttt{interaction} function has been called, the first thing to do is to check whenever the \texttt{file\_name} variable is blank or not. If \texttt{file\_name} is \texttt{None} then a new one is taken and applied from the source code file otherwise the script will continue with the handling of the events.
\begin{python}
if self.file_name is None:
    self.file_name = inspect.getfile(frame)
\end{python}

\subsection{Event Handling}
The first handled event type is the \texttt{call} type. This kind of event is normally happening when the frame of the code is changing and thus is really short. Indeed, it just need to capture the frame name (line 2) and catch the line number (line 3). Nothing else special is handled there.
\begin{python}
if event == 'call':
    self.cur_framename = str(frame.f_code.co_name)
    self.prev_lineno[self.cur_framename] = frame.f_lineno
    self.set_step() # continue
\end{python}

Following, the second event type is the the \texttt{line} type which occurs at each line-break. This event is the most important for the data collection and its operating has to be explained in separated steps. First, the interaction function checks the type of the event and then proceed to extract the line number which is a key information for the user interface. Then for each line, the interpreted objects have to be caught. This is handled by a external function called \texttt{\_filter\_locals} and called with the frame locals in option.
\begin{python}
locals = self._filter_locals(frame.f_locals)
\end{python}

The function itself create first an empty dictionary which will store the name and the value of each local (line 2). The locals starting with a double underscore are ignored and only the specified object are fetched (line 4 to 6). The function returns the \texttt{new\_locals} dictionary to the main \texttt{interaction} function (line 9).

\begin{python}
def _filter_locals(self, local_vars):
    new_locals = {}
    for name, value in list(local_vars.items()):
        if name.startswith('__'):
            continue
        if not isinstance(value, self.instrumented_types):
            continue
        new_locals[name] = [copy.deepcopy(value)]
    return new_locals

\end{python}

Then, the locals are stored in a JSON defaultdict object along with the file name, the frame and the line number. At the end, the JSON dictionary is periodically stored in the database in order to flush the data from the memory and enhance the run-time performances. The population of the database is detailed in the next point.

\begin{python}
if self.total_linenb > self.next_backup:
    self._populate_db()
    self.next_backup += self.next_backup
\end{python}

The handling of the \texttt{line} event is now finished and interaction function continues with the two last types. The \texttt{return} event only occurs at the begging of a file for which we just set the main frame name (line 2) and the \texttt{exception} event happens when there is an error in the code which is printed out in the console (line 6).
\begin{python}
if event == 'return':
    self.cur_framename = '<module>'
    self.set_step()  # continue
if event == 'exception':
    name = frame.f_code.co_name or "<unknown>"
    print("exception in", name, exception)
    self.set_continue()  # continue
\end{python}

\subsection{Trace Ending}

Finally, the data capture model is ended by the \pythoninline{set_quit()} BDB function which was remodeled for writing the last traced lines (line 6).
\begin{python}
def set_quit(self):
    self.stopframe = self.botframe
    self.returnframe = None
    self.quitting = True
    sys.settrace(None)

    if self.json_results:
        if settings.DEBUG:
            print(self.json_results)
        else:
            self._populate_db()
\end{python}



\section{Data model}
The data model is an in-between layer used for the Data capture model and the user interface. Both modules parts will be explained in this section along with the presentation of the data model itself.

\subsection{Definition}
The data model itself evolved a lot during the development and lead to the finale state which will be presented here. This can be observed in the chosen nomenclature, which sometimes does not exactly correspond to the reality. The best example is the use of the substantive "file" in the code which actually describes more an analysis instance or a run than the file itself. Thanks to the MongoDB database engine, it is easy to modify the document structure without any database manipulation in opposition with the tabled nature of SQL engines. This was a great asset which allowed tremendous saving time in the development of the data model since it changed a significant number of times.

To understand the data model it is a good reminder to enumerate what the data capture model is actually capturing. First the data capture model is searching for objects, i.e. integer, string, float variables, and their values. These objects are linked with line number, which are them-self linked with frames. Finally, each frame is owned by a file (or more specifically a run as it was pointed out previously).

Keeping that in mind the different data structures can be considered as documents and defined the following way in Python. This notation is used further for reading from the database. First, the \textit{line} which has a number and some data (objects):
\begin{python}
class Line(EmbeddedDocument):
    lineno = IntField()
    data = DictField()
\end{python}

Secondly, the \textit{frame} which has a name and contains one or many lines :

\begin{python}
class Frame(EmbeddedDocument):
    name = StringField()
    lines = ListField(EmbeddedDocumentField(Line))
\end{python}

Finally, the \textit{file} which as a name, a time-stamp, the content itself (source code) and additionally a revision number gathered from the git repository when available and also the user name of the person who started the analysis. The file contains logically the different frames and the whole is defined this way :
\begin{python}
class File(Document):
    user = StringField()
    revision = StringField()
    filename = StringField()
    timestamp = DateTimeField()
    content = StringField()
    frames = ListField(EmbeddedDocumentField(Frame))
\end{python}

\subsection{Writing to the database}
This part of the database handling is directly implemented in the data capture model along with the capture functionality. The process can be called in two different state of the analyzing phase : 
\begin{itemize}
  \item The program reached the limit of lines and need to flush the gathered data into the database. This occurs inside of the \pythoninline{interaction()} function which has already been described in the previous section. 
  \item The system reached the end of the targeted software and the function \pythoninline{set\_quit()} has been called.
\end{itemize}
Both states induce the call of the \pythoninline{\_populate\_db()} which is constituted of an \pythoninline{if...else} condition. This condition checks whenever it is the first time the system tries to backup the data or not and calls respectively the \pythoninline{\_create\_new\_file()} or the \pythoninline{\_update\_file()} functions (line 3 and 6).
\begin{python}
def _populate_db(self):
    if self.file_id is None:
        self._create_new_file()
        self._clear_cache()
    else:
        self._update_file()
        self._clear_cache()
\end{python}

If the system need to create a new document in the database, as already stated it will call the \pythoninline{\_create\_new\_file()} which is explained here in a simplified and step-by-step version. The complete version of the function includes also a compatibility layer for Python 2 but has been removed here for readability reasons. The first step of the creation of a new entry is to fetch each row of the JSON type dictionary (line 2) where the data has been stored until now and store the data in two different variables (\pythoninline{module\_file} and \pythoninline{frames}).
\begin{python}
def _create_new_file(self):
  for module_file, frames in self.json_results.items():
\end{python}
 
Then, in order to display also the source code in the user interface, the content of the file retrieved (line 1-3) and as all the needed information are already there, the file document type can be created (line=4). The \pythoninline{user} and the \pythoninline{revision} variables are gathered from two function which retrieve the git repository information, but will not be explained in this report.
\begin{python}
file = open(module_file, 'r')
file_content = file.read()
file.close()
item = File(user=self._get_git_username(), revision=self._get_git_revision_short_hash(), filename=module_file, timestamp=datetime.now(), content=file_content)
\end{python}

The next step is to create the \pythoninline{frame} document (line 2) along side with each \pythoninline{line} document belonging to this frame (line 4). Finally, the frame is linked to the file (line 6) and the file can be saved into the database (line 7). Additionally the variable \pythoninline{file\_id}, which was previously defined, is set.
\begin{python}
for name, lines in sorted(frames.items()):
    frame = Frame(name=name)
    for lineno, data in sorted(lines.items()):
        line = Line(lineno = lineno, data = data)
        frame.lines.append(line)
    item.frames.append(frame)
    item.save()
self.file_id = item.id
\end{python}

Now that a first backup has been created in the database for our run, the system will probably have to update the database with the following analyzed lines . With this end in mind, the \pythoninline{\_update\_file()} has been implemented and is presented the same way as the foregoing function. First, as in the previous function the JSON dictionary is looped in order to gather the needed data.
\begin{python}
def _update_file(self):
  for module_file, frames in self.json_results.items():
\end{python}

Then for each frame, the system first checks if it is a new frame or not (line 2) and hence create it in the database (line 3-4). Finally the new analyzed lines are created and saved in the database (line 5-7).
\begin{python}
for name, lines in sorted(frames.items()):
    if not File.objects(id=self.file_id, frames__name=name):
        frame = Frame(name=name)
        File.objects(id=self.file_id).update(push__frames=frame)
    for lineno, data in sorted(lines.items()):
        line = Line(lineno = lineno, data = data)
        File.objects(id=self.file_id, frames__name=name).update(push__frames__S__lines=line)
\end{python}

\subsection{Reading from the database}

Reading from the database exclusively arises in the user interface module. In order to cut down the procedure, the \textit{MongoEngine} library has been chosen. The MongoEngine is a Document-Object Mapper for working with MongoDB from Python. Hence the use of this library gathering the data of a run in the database stands in one line of code.
\begin{python}
file_object = File.objects(id=file_id)
\end{python} 

This line of code gather the complete data of a run, but thanks to the API of MongoEngine it is also possible to retrieve specifically the needed data. If this special options are in the interest of the reader, we suggest to refer directly to the MongoEngine documentation.

\section{User interface}
The user interface is a web application which helps the programmer to review the result of the data capture model. In this section, the focus will be made on the features rather than on the code for different reasons. First, the user interface code is not considered here as the core knowledge of the developed system. Then the complete web application represent almost 1000 lines of code and would stretch considerably out this report with few added value. Finally a non negligible part of the code fulfill a visual and presentation purpose rather than actual features. This is why only some key code section will be showed directly here.

\subsection{Technologies}
The web application is based on the Python web framework \textit{Flask} which is intended to be as lightweight as possible. The flask micro-framework comes with some handy features such as built-in development server and debugger, integrated unit testing support, RESTful request dispatching, or \textit{Jinja2} templating. Flask is normally design to connect with standard SQL databases but with the help of the \textit{flask-mongoengine} extension the process is pretty straightforward.

The shaping of the web-application is indeed done with the help of \textit{HTML/CSS} and \textit{Javascript}. For purposes of standardization, the \textit{Bootstrap} framework came in help which includes also the convenient \textit{JQuery} javascript framework. Bootstrap is a popular HTML, CSS, and JS open-source framework for developing front-end projects on the web. As for Jquery it is a fast, small, and feature-rich JavaScript library which makes things like HTML document traversal and manipulation, event handling, animation, and Ajax much simpler with an easy-to-use API that works across a multitude of browsers.

Some other libraries came also during the development to format the data out of the database. With this aim in mind, the first used library was DataTables which makes formatting data in table a child's play. Additionally the \textit{Highcharts} Javascript library was used to generate every graphs and \textit{pygments} came for highlighting the Python syntax.

\subsection{The cockpit}
The user interface is composed of three main pages. The first one presented here is the index page which simply propose an overview of the different runs stored in the database. This page also act as a cockpit which allows the user to view runs in detail, select runs for comparison and to delete unwanted runs from the database. The runs are searchable and can be sorted by dates, file name, Git revision or user name. It is also possible for the user to configure how many runs should be shown per page.
\begin{figure}[h!]
  \centering
    \includegraphics[width=\textwidth]{figures/jenkins.png}
    \caption{The index page}
    \label{fig:jenkins}
\end{figure}

\subsection{The file reviewer}
Once the user selected a run in the cockpit, he will be directly redirected to a reviewing page as illustrated by the \autoref{fig:viewafile}. 
\begin{figure}[h!]
  \centering
    \includegraphics[width=\textwidth]{figures/jenkins.png}
    \caption{Viewing a file}
    \label{fig:viewafile}
\end{figure}

This page is separated in 4 main sections. First, in the head of the page the user gets some basic information about the run such as the file name, the date of the run or the git revision. Directly underneath, when possible, a plot graph is computed and shows by default all the available objects. The user can directly from the graph choose to hide some variable and the scope of the graph is automatically adapted to the remaining values. Additionally, the chart can also be printed and exported in several different formats. A focused view of the two first sections is depicted in the \autoref{fig:infoandgraph}.
\begin{figure}[h!]
  \centering
    \includegraphics[width=\textwidth]{figures/jenkins.png}
    \caption{Run information and graph}
    \label{fig:infoandgraph}
\end{figure}

A bit further, separated in two vertical columns, the user will find in the left part the complete source code of the analyzed file. The pieces of code which were genuinely analyzed are highlighted in a light yellow color and moreover every line has been numbered and syntactically colored. The line numbers are clickable and open a panel in the left column with some further information about the analyzed objects. Each panel contains a header with the frames name and the line number, and a body with the objects names, their values and a small inline graph resume the evolution of the value. The figure \autoref{sourcepanel} shows in detail these described features. Additionally on the top of the right column some buttons allow to show all the panel, close them or selectively open all panel linked to a frame.

\begin{figure}[h!]
  \centering
    \includegraphics[width=\textwidth]{figures/jenkins.png}
    \caption{Run information and graph}
    \label{fig:sourcepanel}
\end{figure}

\subsection{File comparison}
If needed the user has also the possibility to select several files in the cockpit in order to compare them. This is done by checking the needed runs and clicking the "compare" link at the bottom of the page. Doing so will redirect the user on the start page of the comparison. From this page, each file can be inspected and the user is given the choice of which object he wants to select for the comparison as shown on the \autoref{fig:comparisonstart}.
\begin{figure}[h!]
  \centering
    \includegraphics[width=\textwidth]{figures/jenkins.png}
    \caption{Run information and graph}
    \label{fig:comparisonstart}
\end{figure}

For each selected object, the user has the possibility to see the source file or eventually to unselect it. When he is happy with his selection, graphs can be generated with the triggering of the "Generate graphs" button. For each run a graph will be created an disposed in a way which facilitates comparison as shown on \autoref{fig:generatedgraphs}.
\begin{figure}[h!]
  \centering
    \includegraphics[width=\textwidth]{figures/jenkins.png}
    \caption{Run information and graph}
    \label{fig:generatedgraphs}
\end{figure}


\section{Concluding remarks}
In this chapter, we tried to give an brief but complete insight of the implementation of our solution. It was quite a challenge to summaries 6 monthes of development, over 2000 lines of code (around 32'000 with all libraries included !) in a short and comprehensive chapter.



% Chapter 4

\newglossaryentry{pip}{name=pip, description={Pip Installs Packages is a package management system used to install and manage software packages written in Python}}

\chapter{Installation guide} % Write in your own chapter title
\label{chap:installation}
\lhead{Chapter 4. \emph{Installation}} % Write in your own chapter title to set the page header
\begin{flushright}
\textit{``If Microsoft ever does applications for Linux it means I've won.''} \\ Linus Torvalds
\end{flushright}


In this chapter, the complete installation process of the developed script will be presented.

\section {Setting up the environment}

In order to use the developed tool, it is highly recommended to install it on a system providing a GNU/Linux distribution. The tool might work under Windows or MacOS as the used libraries should all be cross-platform, but the software has never been tested under these platforms.

As the main used language is Python you should install it with the following command :


\section{Use the packaged version}
In order to simplify the installation process, a packaged version has been built and is ready to download on the project's \href{https://github.com/dchenaux/Yoda}{GitHub page}.

The installation process is really straightforward and since it's a \gls{pip} package.

\section{From source code}


\newglossaryentry{md5}{name=MD5, description={The MD5 algorithm is a widely used hash function producing a 128-bit hash value}}
\newglossaryentry{cpu}{name=CPU, description={Central processing unit}}
\newglossaryentry{os}{name=OS, description={Operating system}}
\newglossaryentry{gpu}{name=GPU, description={Graphical processing unit}}
\newglossaryentry{ram}{name=RAM, description={Random-access memory}}


\chapter{Experiments} % Write in your own chapter title
\label{chap:experiments}
\lhead{Chapter 5. \emph{Experiments}} % Write in your own chapter title to set the page header

In this section different type of experiments are conducted in order to test and check the performance of the developed software.

\section{Test script and machine}
In order to conduct the different experiments, we chose a test script available online on \textit{TheAlgorithms} \href{https://github.com/TheAlgorithms/Python/blob/master/hashes/md5.py}{GitHub page}. We selected this test script because \gls{md5} checking is a really common task and generates many numerical values for which our solution is the most intended. 

Concerning the testing machine itself, we chose to run the tests directly on our personal laptop because it correspond to the usage the system is intended for : a personal debugging tool. The specifications of the machine are the following ones :
\begin{itemize}
  \item Lenovo Thinkpad T460p
  \item \gls{cpu} : Intel Core i7-6700HQ @ 2.60GHz x 8
  \item \gls{os} : Fedora 25 64bits
  \item \gls{gpu} : Intel HD Graphics 530
  \item \gls{ram} : 15.1Gio
\end{itemize}

\section{Objectives}
For the experiments, we want here to test two aspects of our system : the memory usage and the runtime. To be able to measure these two parameters, we use a Python tool called \textit{Memory profiler}. This tool is for monitoring memory consumption of a process as well as line-by-line analysis of memory consumption for python programs. It allows also to plot memory consumption as a function of time and measure the execution time of the target script. With these extracted parameters, the tool is also capable to plot graphs. Our aim is to verify if dynamic analysis induces overhead with the help of the memory profiler and our dynamic analysis solution.

\section{Performances}
\subsection{Test script}
The first necessary step in order to conduct our experiments correctly, is to create a reference run from which we will be able to compare our results.
In this idea, we profiled the memory consumption and the runtime of the test script without plugging in our system. The \autoref{fig:memumd5} illustrates the results of the Memory profiler: the test script uses a total of 13.45MiB of memory for a runtime of 0.1s.
\begin{figure}[h!]
  \centering
    \includegraphics[width=\textwidth]{figures/experiments_figure_md5.png}
    \caption{Memory usage and runtime of the test script}
    \label{fig:memumd5}
\end{figure}
\subsection{Data capture}
The first process in our solution which could induce overhead is the data capture process and therefore, we want here to test its performance. In order to exclusively determine the memory usage and runtime of the data capture process, we activated the debug mode of the system to avoid the database writing process. 

The \autoref{fig:memuyoda} presents the results of the memory profiler analysis. As we supposed, the process is inducing overhead. The execution time is now of 0.6s which represents a multiplication by 6 compared to the reference test. Concerning the memory the script now needs 27.6MiB of RAM which is more than the double of the original run.
\begin{figure}[h!]
  \centering
    \includegraphics[width=\textwidth]{figures/experiments_figure_yoda.png}
    \caption{Memory usage and runtime with the Yoda system activated}
    \label{fig:memuyoda}
\end{figure}

These results could be seen as deceiving but in fact are inevitable because of the nature of our solution. Indeed, because we want to track the value of each variable at each line, the values are not overridden as in the normal run of the script. Instead, for each value, we store in the memory a copy of the variable and its data.

\subsection{Database writing}
The second process of interest which we want to test here is the database writing process. By activating this phase in the tool, we want to see if it also induces an overhead. 

As it is shown in the \autoref{fig:memudb}, the introduction of the data writing process in our test does not induce any significant extra memory usage which is now at the maximum around 29MiB. Nonetheless, the writing process induce a runtime overhead of 0.5s to now reach a total of 1.21s.
\begin{figure}[h!]
  \centering
    \includegraphics[width=\textwidth]{figures/experiments_figure_db.png}
    \caption{Memory usage and runtime with the database writing}
    \label{fig:memudb}
\end{figure}

\section{Concluding remarks}

In the \autoref{chap:relatedwork}, we already introduced some limitation of dynamic analysis tools, which is often heavy runtime overhead. We showed in the experiments that our solution is not an exception and induces even on a small script a heavier memory consumption and a significant runtime overhead. An idea to optimize the process could be asynchronously writing via a message queue.

It has to be pointed out that the memory profiler is not an exact tool. Indeed, according to \cite{Pedregos2016}, this module gets the memory consumption by querying the operating system kernel about the amount of memory the current process has allocated, which might be slightly different from the amount of memory that is actually used by the Python interpreter. Also, because of how the garbage collector works in Python the result might be different between platforms and even between runs.


\chapter{Conclusion} % Write in your own chapter title
\label{chap:conclusion}
\lhead{Chapter 6. \emph{Conclusion}} % Write in your own chapter title to set the page header

In this final chapter, we summarize the work done in this master thesis and give complementary remarks along with potential improvements that came into mind during the development and then the writing of the report.

\section{Final remarks}
The main goal of the thesis was to develop a system which could pursue effective code maintenance with continuous data collection. This report is the summary of 6 months of development and in here we additionally presented related works in order to build a theoretical background. Our proposed solution is explained in a dedicated chapter along with a handy installation guide. The explanations were focused on the core features of our system because we did not want to overwhelm the reader with too much technical aspects. Indeed, even if the coding of the user interface was around half of the coding time, we only presented the final features without spending time about its programming implementation.  Additionally, we conducted some experiments on the developed applications to find out if it engendered some drawbacks. During the development, we thought also about some future work which could not be included in the scope of a master thesis. These points are exposed in the next section.

\section{Personal remarks}

On a personal point of view, this master thesis was the second experience in the development of a complete high technical subject. But even though, I had already the chance to develop a bin packing system for my Bachelor Thesis, this was a completely new challenge in a absolutely unfamiliar field. As the development was cut in two half, because I was in Austria for an internship during the summer, I also lost some time to step back into the code after this long pause. On the other hand, this brought a fresh view on the code and some parts that seemed really good before were rewritten in a more clever way.

Also, the use of Python was a new challenge for me as I was more used to develop in PHP for this kind of application and therefore I had to learn to know and use all the different libraries. The choice of MongoDB was also a discovery as I was more used to work with relational databases.

Finally, to conclude this work, I learned the hard way that programmers may plan as well as they want the available time, but there will always be unexpected events. Particularly on this kind of research field, it was quite difficult for me to estimate how much time could be needed to develop an idea. I can clearly remember that after 2 months of development I already had the felling that I was close to the end. This is something I want to take with me for my future projects.

\section{Future work}
As the developed application is only a proof-to-concept system, there is plenty of room for future improvements. In this final section, we want to propose the reader some topics and features that came into mind during the development and the final experiments. These improvements could lead to a software which could be used by final users. Following, in order of importance, some of the further work we could think about :
\begin{itemize}
  \item At the time, the data capture model has some issues with complex data retrieval. Indeed, it is more than likely that the system will throw an error when the analyzed script opens large text files in order to work with them. Our insight is that there is a conflict between the JSON way of the MongoDB database and the pythonic way of storing dictionaries ;
  \item Currently, the system was focused on numeric variables, even if it is capable of retrieving strings. We would suggest some enhancement in that way by first removing the string variables from the plot charts and then perhaps find an another way to represent them. One idea could be to create a kind of tree mapping of the different values ;
  \item The plot graphs definitely need a zoom functionality. In deed, we observed that with scripts which can generate a big amount of values it is increasingly hard to use the graphs ; 
  \item The following point is something we already discussed at the beginning of the coding phase and it is still an open point. In the actual proposed solution, all the values are stored in the database. In the example of a looping variable which could range from 1 to 1 million, is it really useful to store every value in between ? We think that there could be a simpler way to store such data ;
  \item A last point concerning the user interface itself, thanks to some new framework like \textit{Electron}, it would be pretty easy to adapt it to a standalone application.
\end{itemize}
  



%% ----------------------------------------------------------------
% Now begin the Appendices, including them as separate files


\appendix % Cue to tell LaTeX that the following 'chapters' are Appendices


%\input{./appendices/appendixA}	% Appendix Title

\printglossaries

\chapter{License of the software}
\label{appendix:license}

Copyright (c) 2016 DAVID CHENAUX

Permission is hereby granted, free of charge, to any person obtaining a copy of this software and associated documentation files (the "Software"), to deal in the Software without restriction, including without limitation the rights to use, copy, modify, merge, publish, distribute, sublicense, and/or sell copies of the Software, and to permit persons to whom the Software is furnished to do so, subject to the following conditions:

The above copyright notice and this permission notice shall be included in all copies or substantial portions of the Software.

THE SOFTWARE IS PROVIDED "AS IS", WITHOUT WARRANTY OF ANY KIND, EXPRESS OR IMPLIED, INCLUDING BUT NOT LIMITED TO THE WARRANTIES OF MERCHANTABILITY, FITNESS FOR A PARTICULAR PURPOSE AND NONINFRINGEMENT. IN NO EVENT SHALL THE AUTHORS OR COPYRIGHT HOLDERS BE LIABLE FOR ANY CLAIM, DAMAGES OR OTHER LIABILITY, WHETHER IN AN ACTION OF CONTRACT, TORT OR OTHERWISE, ARISING FROM, OUT OF OR IN CONNECTION WITH THE SOFTWARE OR THE USE OR OTHER DEALINGS IN THE SOFTWARE.


\clearpage

\addtocontents{toc}{\vspace{2em}}  % Add a gap in the Contents, for aesthetics
\backmatter

%% ----------------------------------------------------------------
\label{Bibliography}
\lhead{\emph{Bibliography}}  % Change the left side page header to "Bibliography"
\bibliographystyle{plainnat}  % Use the "unsrtnat" BibTeX style for formatting the Bibliography
\bibliography{thesis}  % The references (bibliography) information are stored in the file named "Bibliography.bib"
\clearpage
%% ----------------

\pagestyle{empty}
\label{Declaration}
\includepdf[pages=-, offset=75 -75]{appendices/declaration.pdf}

\clearpage


%% ----------------------------------------------------------------
%\cleardoublepage
%\phantomsection
%\addcontentsline{toc}{chapter}{Index}
%\label{Index}
%\printindex


\end{document}  % The End
%% ----------------------------------------------------------------
